\section{Introduction}
The Chimbuko framework captures, analyzes and visualizes performance metrics for complex scientific workflows that enables investigations into co-design tradeoffs for data reduction on extreme-scale machines.
Chimbuko helps comparing different runs at high and low levels of metric granularity. Chimbuko provides this capability in both offline and online (in-situ) modes. Because capturing performance metrics can quickly escalate in volume and provenance can be highly verbose, Chimbuko includes a data analysis module for data reduction. 

Chimbuko enables co-design studies by allowing scientific applications and workflows to profile their execution patterns on traditional and heterogeneous architectures.  The detailed metrics and information (a.k.a. provenance of the execution) once extracted are reduced and visualized in Chimbuko analysis and visualization modules, thus providing application developers insights into the behavior of their code at scale for the purpose of code optimization and new code development.  Using provenance, Chimbuko provides these insights for both single applications and complex orchestrated workflows, that are becoming more prevalent to organize complex code execution.

Currently, Chimbuko provides performance visualization and data analysis for offline mode.  For the second year, the framework will provide performance visualization and data analysis helped by provenance information for  online performance analysis. Prescriptive provenance will be used to assist the analysis in selecting features of interest both in scientific results and performance metrics.
	
