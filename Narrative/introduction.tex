\section{Introduction}
The Chimbuko framework captures, analyzes and visualizes performance metrics for complex scientific workflows and relates these metrics to the context of their execution on extreme-scale machines. The purpose of Chimbuko is to enable empirical studies of performance analysis for a software or a workflow during a development phase or in different computational environments. Chimbuko enables the comparison of different runs at high and low levels of metric granularity. Chimbuko provides this capability in both offline and online (in-situ) modes. Because capturing performance metrics can quickly escalate in volume and provenance can be highly verbose, Chimbuko includes a data reduction module. 

Currently, Chimbuko provides performance visualization and data analysis for offline mode.  For the second year, the framework will provide performance visualization and data analysis for online mode.		
