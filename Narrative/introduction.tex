\section{Introduction}
The Chimbuko framework captures, analyzes, and visualizes performance metrics for complex scientific workflows that enables investigations into co-design tradeoffs for online data analysis and reduction on extreme-scale machines.
Chimbuko helps to compare different runs at high and low levels of metric granularity. Chimbuko provides this capability in both offline and online (in-situ) modes. Because capturing performance metrics can quickly escalate in volume and provenance can be highly verbose, Chimbuko includes a data analysis module for data reduction. 

Chimbuko enables co-design studies by allowing scientific applications and workflows to profile their execution patterns on traditional and heterogeneous architectures.  The detailed metrics and information (a.k.a. provenance of the execution) are extracted and then reduced and visualized by the Chimbuko analysis and visualization modules, thus providing insights into the behavior of the codes and workflows at scale.
%% for the purpose of code optimization and new code development.  
Using provenance, Chimbuko provides these insights for both single applications and the complex orchestrated workflows that are becoming more prevalent to organize complex code execution.

Currently, Chimbuko provides performance visualization and data analysis in an offline mode.  The next release of the framework will provide performance visualization and data analysis, helped by provenance information, for online performance analysis. Prescriptive provenance will assist in selecting features of interest in both the scientific results and the performance metrics.

