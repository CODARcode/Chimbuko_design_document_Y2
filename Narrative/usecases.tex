\section{Use cases}
We have following use cases:

\subsection{NWCHEM}
In the NWChemEx project we will be studying processes involving transmembrane proteins as well as zeolite catalysts. The processes of interest require the calculation of free energies and the dynamics of the molecular structures. In addition, to simulate realistic molecular environments, the molecular structures will have at least 100,000 atoms, different regions may be evaluated at different levels of approximation, and the simulations will work with time steps of about 1 femtosecond.  Hence to sample enough of the phase-space a larger number of time steps will be evaluated (think of 1 million time steps). The way the simulations will be run requires the evaluation and forces as well as updating the molecular structures. While these calculations are executed, the statistics needed for the free energies are collected. In addition selected structures along the trajectory will be stored, so that additional properties for those can be calculated. These additional properties will facilitate comparing the calculated trajectories against experimental observations.  Hence overall there will be a large number of cores calculating the energies and forces alongside a small number of cores analyzing the results and storing selected time steps for more detailed simulations. 

%\subsubsection{Requirements for the Online Performance Analysis}
There will be a large number of parameters that define the total amount of work in particular parts of the simulation, and different amounts of work will change the optimal work distribution. An important aspect will be that the performance characteristics need to be recorded in a way that can be compared against prior simulations to establish the figures of merit of the development. This requires capturing some base characteristics that are always the same. 
For specific performance optimizations it may be necessary to capture the performance of specific parts of the code, depending for example, on the functionality of interest, or on the characteristics of the data distribution, or on the granularity of the tensor blocks and associated task sizes. Dependent on these kinds of characteristics the data collection may be turned on or off. The shear volume of the data expected requires the analysis to be performed online

In order to extract and analyze interesting events both for performance and scientific results with in-situ analysis, prescriptive provenance will be extracted by Chimbuko.  This provenance extraction is needed to provide data analysis with execution metrics used to build training sets, classify features of interest, and select relevant events. 


\subsection{QCD}
Last year we experimented with using TAU for a single benchmark calculation (single application) (https://github.com/meifeng/Example-LatticeQCD-With-TAU), which is not representative of the typical production lattice QCD simulations.  These are orchestrated in workflows and consist of several more complex components, including propagator calculations and contractions. Provenance needs to be extracted and persisted as workflows exhibit complex interdependencies at runtime to enable diagnosis of latencies and bottlenecks for code development. The performances of these calculations are often limited by the data transfer rates, both intra-node (depending on node architecture) and internode (mpi). IO can also be a limiting factor for some of the algorithms in the LQCD calculations. This year, while the LQCD code is undergoing constant development, we will look into using Chimbuko to get a more comprehensive understanding of the performance bottlenecks, to guide our development, and to provide feedback to the tool developers. As the code is evolving to adapt to the pre-exascale architectures such as Summit, we will target to have the first comprehensive study of the various performance metrics related to lattice QCD simulations at the beginning of FY19. 

\subsection {LAMMPS}

LAMMPS (Large Scale Atomic/Molecular Massively Parallel Simulator)  is widely used Molecular Dynamics simulation engine that studies materials science and adopts MPI for parallel communication. The use case of LAMMPS is a workflow that is composed of three components, the LAMMPS application, the $Voro++$ was (analytics for LAMMPS) and an ADIOS based parallel data writer component (stage\_write). The workflow is configurable, so LAMMPS can communicate with $Voro++$ either directly or through stage\_wtite for adopting different IO strategies. The flexibility of the LAMMPS use case can help explore performance tradeoffs on different machines.

\subsection{Fusion}
At $SC17$, some members of the CODAR team and other ECP projects collaborated to couple two fusion simulations, each running the XGC application on Titan at ORNL.  One simulation was simulating the core of the plasma of a fusion reactor, the other was simulating the edge. In this coupling, ADIOS was used for data exchange at every timestep.  During execution, TAU was measuring both XGC simulations, and aggregating the performance data at runtime over SOS at each ADIOS data exchange. The performance data measurement was limited, just to MPI and ADIOS events, as well as overall application performance.  The performance data was analyzed at runtime, extracting out MPI and PMI coordinate information, so that the performance data could be visualized in 3D at runtime using the PMI coordinates.  Future versions of this code coupling will replace one instance of XGC with the GENE simulation.  
In coupling, we need to analyze various aspects of the communication pattern not only between XGC and GENE but also inter communications of each application in order to decide optimal placement of coupling processes.  Prescriptive provenance specifies detailed metrics of this communication. 
The performance data extraction (for the time being) has also been simplified to only produce 2D scatterplots of memory consumption for each process, as well as FLOPS for each process (in order to provide a ``dashboard'' for runtime observation).  This simulation could benefit from Chimbuko integration by introducing anomaly detection and richer visualization than is currently provided. 
